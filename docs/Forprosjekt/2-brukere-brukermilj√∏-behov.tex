\section{Brukere, brukermiljø og behov}

\subsection{Brukere}  

Teknisk bruker av applikasjonen er da team for kvalitetsregister ved HN-IKT. Sluttbrukere som får levert ut kodene er registrene nevnt i introduksjon.\\ 

\subsection{Brukermiljø}  

Applikasjonen blir en intern del av kvalitetsregister mikrotjeneste plattformen, da den i hovedsak erstatter det som i dag er en tungvint og manuell prosess. Dette vil igjen påvirke brukerne av kvalitetsregistrene som vil oppleve at kvalitetsregistrene stadig er oppdatert med nyeste versjoner av kode verkene.

Sluttbrukerne vil ikke oppleve endringer på bakgrunn av dette arbeidet i daglig bruk av registeret, i de tilfeller der kodeverk er oppdaterte.

Den merkbare forskjellen er at det ikke vil være forsinkelse fra publisering av ny data til den er tilgjengelig i register-frontenden.
 
\begin{landscape}
\subsection{Sammendrag av brukerens behov} 
\begin{table}[H]
\centering
\begin{tabular}{|c|c|c|c|c|} 
\hline
Behov & Prioritet & Påvirker & Dagens Løsning & Foreslått løsning\\
\hline\hline
Automatisk datafangst & Høy & Alle brukere & Manuell innhenting & Hente fra eksterne kilder via web API\\
\hline
Versonstyring & Høy & Alle brukere & Manuelt ved lagring i database & Automatisk ved lagring i database\\
\hline
Tilgjengeligjøring av oppdatert data & Høy & Alle brukere & Manuell prosess & Automatisk ved innhenting av ny data\\
\hline
Mellomlagring i registeret & medium & Alle brukere & Lagres i MySql & Lagre in-memory. f.eks REDIS\\
\hline
Tilgjengeligjøring av dokumenter & medium & Alle brukere & Lagres i ressurs mapper & Lagre versjonert i database\\
\hline
\end{tabular}
\caption{Sammendrag av brukerens behov}
\label{table}
\end{table}
\end{landscape}
