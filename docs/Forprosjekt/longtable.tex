%TABLE - PART - ONE
\begin{landscape}
\subsubsection{Risikoanalyse}
\begin{longtable}[H]
\begin{tabular}{|c|p{4cm}|c|c|c|c|p{4cm}|p{4cm}|}    
\hline
Nr. & Beskrivelse av hendelsen & Påvirker & Sannsynlighet & Konsekvens & Risikonivå & Forebyggende tiltak & Korrigerende tiltak  \\
\hline\hline
1 & Prosjektet ble ikke startet innen den gitte tidsrammen & Prosjekt & 2 & 2 & 4 & Oppstartsmøte. Velge kommunikasjonsverktøy. Etablere arbeidsplan og -rutiner. & Ved behov sette litt ekstra tid for å starte å jobbe med prosjektet i tide.   \\
\hline
2 & Prosjektet ble ikke fullført innen den gitte tidsrammen & Prosjekt & 3 & 4 & 12 & Aktivt bruk av smidige utviklingsmetoder i løpet av hele prosjektperioden. Ha kontroll over utførelsen av ulike aktiviteter. God kommunikasjon mellom gruppemedlemmer og veiledere.  & Ta kontakt med oppdragsgiver og veiledere for å justere arbeidsplan. Diskutere mulighet for gjennomføring av manglende aktiviteter i et annet prosjekt.   \\
\hline
3 & Sykdom eller skade i en gruppe & Person / Prosjekt & 4 & 2 & 8 & Unngå alvorlig sykdom og redusere aktiviteter, som kan føre til skade. & De andre gruppemedlemmer hjelper ved behov  \\
\hline
4 & Dobbeltarbeid & Person / Prosjekt & 2 & 2 & 4 & God kommunikasjon mellom gruppemedlemmer. Utarbeide og følge arbeidsplan. Aktivt bruk av Jira Software (Scrum).    & Gjøre nødvendige justeringer i arbeidsplanen. Utføre de kritiske aktivitetene. Omprioritering av aktiviteter ved behov.  \\
\hline
\pagebreak
%\end{tabular}
%\caption{Risikotabell som viser mulige uønskede hendelser, grad av sannsynlighet, grad av konsekvens, beregnet risikonivå og mulige tiltak \cite{4-forelesning-risikonalyse}.}
%\label{table}
%\end{table}
%\end{landscape}

%TABLE - PART - ONE
%\begin{landscape}
%\begin{table}[H]
%\begin{tabular}{|c|p{4cm}|c|c|c|c|p{4cm}|p{4cm}|}   
%\hline

\pagebreak

Nr. & Beskrivelse av hendelsen & Påvirker & Sannsynlighet & Konsekvens & Risikonivå & Forebyggende tiltak & Korrigerende tiltak  \\
\hline\hline
5 & Dårlig arbeidsmiljø i en gruppe / Konflikter & Person / Prosjekt & 1 & 4 & 4 & Fordeling av roller og ansvarsområder i en gruppe. God kommunikasjon for å forebygge mulige konfliktsituasjoner.  & Ved oppstått misforståelse prøve å finne årsaken til dette. Etterpå finne en generell løsning til problemet. For eksempel, gjøre endringer i ansvarsområdet, justere arbeidsmengde osv.  \\
\hline
6 & Tap av rapport, kildekode, dokumentasjon ol. & Produkt / Prosjekt & 2 & 5 & 10 & Dokumentasjon og kildekode skal lagres både lokalt hos alle gruppemedlemmer og på de valgte skytjenester som GitHub, Overleaf osv.  & Prøve å gjenopprette tapt dokumentasjon og kildekode. Ved behov starte på nytt fra siste checkpoint.  \\
\hline
7 & Ønsket funksjonalitet blir ikke implementert & Produkt / Prosjekt & 2 & 4 & 8 & Under planlegging av prosjektarbeid bør gruppe sette realistiske mål over funksjonalitet som skal bli implementert i applikasjonen.   & Gjøre endringer i prosjektplanen. Prøve å finne alternative løsninger. Prioritering av oppgaver.   \\
\hline
\end{tabular}
\caption{Risikotabell som viser mulige uønskede hendelser, grad av sannsynlighet, grad av konsekvens, beregnet risikonivå og mulige tiltak \cite{4-forelesning-risikonalyse}.}
\label{table}
\end{longtable}
\end{landscape}